\documentclass[a4paper, 12pt]{article}

\usepackage[T2A]{fontenc}			% кодировка
\usepackage[utf8]{inputenc}			% кодировка исходного текста
\usepackage[english,russian]{babel}	% локализация и переносы

\usepackage{amsmath}
\usepackage{icomma} 				% "Умная" запятая
\usepackage{amsfonts}

\sloppy
\usepackage[T2A]{fontenc}
\usepackage{amsmath}
\usepackage{graphicx}
\graphicspath{{pictures/}}
\DeclareGraphicsExtensions{.pdf,.png,.jpg}

\newcommand{\ip}[2]{(#1, #2)}

\title{Д/з - 1}
\author{Чистяков Глеб, группа 167}
\date{\today}

\begin{document}
	
	\maketitle
	
	\textbf{№1} \\
	
	Докажем, что формула $m \circ n = mn - m - n + 2$ задает бинарную операцию на множестве $(\mathbb{Q}\setminus\{1\}, \circ)$:
	
	В $m \circ n$ используются лишь операции $+, -, \times \Rightarrow m \circ n \in \mathbb{Q}$
	Предположим, что $m \circ n = 1 \Rightarrow mn - m - n + 2 = 1 \Leftrightarrow m(n - 1) = n - 1 \Rightarrow m = \frac{n - 1}{n - 1} \Rightarrow n \neq 1$, тогда $m = 1$, но $m, n \in \mathbb{Q}\setminus\{1\}$, а следовательно, наше предположение неверно. Значит $m \circ n \neq 1$ и $m \circ n \in (\mathbb{Q}\setminus\{1\}, \circ)$.
	
	Таким образом, формула $m \circ n = mn - m - n + 2$ задает бинарную операцию на множестве $(\mathbb{Q}\setminus\{1\}, \circ)$. \\
	
	Докажем, что $(\mathbb{Q}\setminus\{1\}, \circ)$ является группой: \\
	1) Проверим $(a \circ b) \circ с = a \circ (b \circ c)$:
	\begin{center}
		$(a \circ b) \circ с = (ab - a - b + 2) \circ c =$ \\
		$= (abc - ac - bc + 2c) - (ab - a - b + 2) - c + 2 =$ \\
		$= abc - ab - ac - bc + a + b + c =$ \\
		$= (abc - ab - ac + 2a) - a - (bc - b - c + 2) + 2 =$ \\
		$a \circ (bc - b - c + 2) = a \circ (b \circ c)$ \\
	\end{center}
	2) Нейтральный элемент: \\
 	Пусть $a$ - какой-то элемент, тогда рассмотрим такой элемент $e$, что $a \circ e = e \circ a = a$:
	\begin{center}
	$a \circ e = ae - a - e + 2 = a \Leftrightarrow e(a - 1) = 2a - 2 \Rightarrow e = 2$ \\
	$e \circ a = ea - e - a + 2 = a \Leftrightarrow e(a - 1) = 2a - 2 \Rightarrow e = 2$
	\end{center}
	$\Longrightarrow e = 2$ - нейтральный элемент. \\
	\newline
	\newline
	\newline
	3) Обратный элемент: \\
	Пусть $a$ - какой-то элемент, тогда рассмотрим $a^{-1}: a \circ a^{-1} = a^{-1} \circ a= 2$, то есть
	\begin{center}
		$a \circ a^{-1} = aa^{-1} - a - a^{-1} = 2 \Leftrightarrow a^{-1}(a - 1) = a \Rightarrow a^{-1} = \dfrac{a}{a - 1}$ \\
		$a^{-1} \circ a = a^{-1}a - a^{-1} - a = 2 \Leftrightarrow a^{-1}(a - 1) = a \Rightarrow a^{-1} = \dfrac{a}{a - 1}$ \\
	\end{center}
	$\Longrightarrow a^{-1} = \dfrac{a}{a - 1}$ - обратный элемент. \\
	
	Таким образом, $(\mathbb{Q}\setminus\{1\}, \circ)$ является группой.
	\newline

	\textbf{№2} \\
	
	Запишем таблицу, где каждому $x \in (\mathbb{Z}_{12}, +)$ соответствует $ord(x)$: \\
	
	\begin{tabular}{|c|rrrrrrrrrrrr|} \hline
		$\mathbb{Z}_{12}$ & 0 & 1 & 2 & 3 & 4 & 5 & 6 & 7 & 8 & 9 & 10 & 11 \\ \hline
		Порядок & 1 & 12 & 6 & 4 & 3 & 12 & 2 & 12 & 3 & 4 & 6 & 12 \\ \hline
	\end{tabular}
	\newline \\
	
	\textbf{№3} \\
	
	Подгруппой группы $G = (\mathbb{Z}_{12}, +)$ назовем такое $H$, для которого:\\
	1) $0 \in H$ \\
	2) $a \in H \Rightarrow a^{-1} \in H$ \\
	3) $a, b \in H \Rightarrow a + b \in H$ \\
	
	Несобственные подгруппы: \\
	\{0\} \\
	\{0, 1, 2, 3, 4, 5, 6, 7, 8, 9, 10, 11\} \\
	
	Собственные подгруппы: \\
	
	Можно заметить, что наша группа $G$ является циклической (так как имеет циклическую подгруппу $<1> = G$). Тогда докажем такое утверждение: \\
	\textit{Каждая подгруппа циклической группы - циклическая.}
	
	Пусть $G = <g_0>, g_0^n \in H, n \in \mathbb{N}$ - наименьшее. $\exists k: \forall a \in <g_0^n> a = (g_0^n)^k \in H \Rightarrow <g_0^n> \subseteq H$.
	\newline
	\newline
	
	Все элементы группы $G$ с образующей $g_0$ представимы в виде $ g_0^n \in H $, где $n \in \mathbb{N}$ - наименьшее. Тогда любой элемент $g \in H$ можно выразить, как $g = g_0^m, m = qn + r, 0 \leq r < n \Rightarrow g = g_0^m = g_0^{qn + r} = (g_0^n)^qg_0^r \Rightarrow g_0^r = (g_0^n)^{-q}g \Rightarrow r = 0 \Rightarrow g = (g_0^n)^q \Rightarrow <g_0^n> \supseteq H$.
	
	Таким образом, $H$ - циклическая, с образующей $<g_0^n>$.\\
	
	Тогда переберем все циклические подгруппы: \\
	$<2> = \{0, 2, 4, 6, 8, 10\}$ \\
	$<3> = \{0, 3, 6, 9\}$ \\
	$<4> = \{0, 4, 8\}$ \\
	$<5> = \{0, 5, 10\}$ \\
	$<6> = \{0, 6\}$ \\
	$<7> = \{0, 7\}$ \\
	$<8> = \{0, 8\}$ \\
	$<9> = \{0, 9\}$ \\
	$<10> = \{0, 10\}$ \\
	$<11> = \{0, 11\}$ \\
	
	Из них нам подходят только $<2>, <3>, <4>, <6>$
	\newline
	
	\textbf{№4} \\
	
	Для начала рассмотрим группу, порядки элементов которых конечны. В этом случае количество конечных циклических подгрупп, очевидно, будет бесконечным. Если же сушествует элемент с бесконечным порядком, тогда будет существовать бесконечная циклическая подгруппа, порождаемоя этим элементом, включающая бесконечное множество циклических подгрупп. Таким образом, любая бесконечная группа сожержит бесконечное число подгрупп.
	
\end{document}
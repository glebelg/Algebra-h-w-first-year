\documentclass[a4paper, 12pt]{article}

\usepackage[T2A]{fontenc}			% кодировка
\usepackage[utf8]{inputenc}			% кодировка исходного текста
\usepackage[english,russian]{babel}	% локализация и переносы

\usepackage{amsmath}
\usepackage{icomma} 				% "Умная" запятая
\usepackage{amsfonts}

\sloppy
\usepackage[T2A]{fontenc}
\usepackage{amsmath}
\usepackage{graphicx}
\graphicspath{{pictures/}}
\DeclareGraphicsExtensions{.pdf,.png,.jpg}

\newcommand{\ip}[2]{(#1, #2)}

\title{Д/з - 4}
\author{Чистяков Глеб, группа 167}
\date{\today}

\begin{document}
	
	\maketitle
	
	\textbf{№1} \\
	
	$G$ - группа всех диагональных матриц в $GL_3(\mathbb R)$ и $X - \mathbb R^3$. \\
	1) $orb(x) = \{gx \arrowvert \forall g \in G\} \subseteq X, x \in X$. Пусть 
	$g = \begin{pmatrix}
			a & 0 & 0\\
			0 & b & 0\\
			0 & 0 & c  
	\end{pmatrix}, a, b, c \in \mathbb R \setminus \{0\}$ и 
	$x = \begin{pmatrix} d \\ e \\ f \\ \end{pmatrix}, d, e, f \in \mathbb R \Rightarrow 
	gx = \begin{pmatrix}
	a & 0 & 0\\
	0 & b & 0\\
	0 & 0 & c  
	\end{pmatrix} \cdot
	\begin{pmatrix} d \\ e \\ f \\ \end{pmatrix} = 
	\begin{pmatrix}	ad \\ be \\ cf \\ \end{pmatrix} \Rightarrow$ орбиты будут иметь вид: \\
	
	$\begin{pmatrix} ad \\ be \\cf \\ \end{pmatrix},
	\begin{pmatrix} 0 \\ be \\cf \\ \end{pmatrix},
	\begin{pmatrix} ad \\ 0 \\cf \\ \end{pmatrix},
	\begin{pmatrix} ad \\ be \\0 \\ \end{pmatrix},
	\begin{pmatrix} 0 \\ 0 \\ cf \\ \end{pmatrix},
	\begin{pmatrix} ad \\ 0 \\ 0 \\ \end{pmatrix},
	\begin{pmatrix} 0 \\ be \\ 0 \\ \end{pmatrix},
	\begin{pmatrix} 0 \\ 0 \\ 0 \\ \end{pmatrix}$ \\
	
	2) $st(x) = \{g \in G \arrowvert gx = x\}, x \in X \Rightarrow 
	gx = \begin{pmatrix}
	a & 0 & 0\\
	0 & b & 0\\
	0 & 0 & c  
	\end{pmatrix} \cdot
	\begin{pmatrix} d \\ e \\ f \\ \end{pmatrix} = 
	\begin{pmatrix} ad \\ be \\ cf \\ \end{pmatrix} =
	\begin{pmatrix} d \\ e \\ f \\ \end{pmatrix} \Rightarrow$
	$\begin{cases}
	ad = d \text{$\Rightarrow a = 1$, при $d \neq 0$ и $a$ - любое, иначе} \\
	be = e \text{$\Rightarrow b = 1$, при $e \neq 0$ и $b$ - любое, иначе} \\
	cf = f \text{$\Rightarrow c = 1$, при $f \neq 0$ и $c$ - любое, иначе} \\
	\end{cases}$ \\
	Таким образом
	$st(x) = \left\{g = \begin{pmatrix}
	a & 0 & 0\\
	0 & b & 0\\
	0 & 0 & c  
	\end{pmatrix}
	\begin{tabular}{|r}
	\text{$a = 1$, при $d \neq 0$, иначе любое} \\
	\text{$b = 1$, при $e \neq 0$, иначе любое} \\
	\text{$c = 1$, при $f \neq 0$, иначе любое} \\
	\end{tabular}\right\}$ \\

	\textbf{№2} \\
	
	$G$ - группа всех верхнетреугольных матриц в $SL_2(\mathbb R)$. Так как мы ищем классы сопряженности, рассмотрим действие сопряжения $(h, g) \rightarrow hgh^{-1}$. Пусть
	$h = \begin{pmatrix}
		a & b \\
		0 & c  
	\end{pmatrix}$ и
	$g = \begin{pmatrix}
	d & e \\
	0 & f  
	\end{pmatrix}$, тогда \\ $hgh^{-1} =
	\begin{pmatrix}
	a & b \\
	0 & c  
	\end{pmatrix} \cdot 
	\begin{pmatrix}
	d & e \\
	0 & f  
	\end{pmatrix} \cdot
	\begin{pmatrix}
	c & -b \\
	0 & a  
	\end{pmatrix} =
	\begin{pmatrix}
	adc & -adb + a^2e + abf \\
	0 & acf 
	\end{pmatrix} = \\ =
	\begin{pmatrix}
	d & -adb + a^2e + abf \\
	0 & f
	\end{pmatrix}$ \\
	Тогда 1) $d = f = \pm1$, $hgh^{-1} =
	\begin{pmatrix}
		d & a^2e \\
		0 & f
	\end{pmatrix}$ \\
	2) $d \neq f$, $hgh^{-1} =
	\begin{pmatrix}
	d & -adb + a^2e + abf \\
	0 & f
	\end{pmatrix}$ \\
	
	\textbf{№3} \\
	
	Имеется группа $S_4$. Пусть $\sigma = (1, 2, 3, 4) \in S_4$. \\
	$st(\sigma) = \{\pi \in S_4 \arrowvert \pi\sigma \pi^{-1} = \sigma\}$, тогда $\pi\sigma \pi^{-1} = \sigma  \Leftrightarrow \pi\sigma = \sigma \pi \Rightarrow$ расcмотрим на этом элементы подстановки: \\
	для 1: $\pi(\sigma(1)) = \pi(2)$, но также $\pi(\sigma(1)) = \sigma(\pi(1)) \Rightarrow \sigma(\pi(1)) = \pi(2)$, аналогично для 2: $\sigma(\pi(2)) = \pi(3)$ \\
	для 3: $\sigma(\pi(3)) = \pi(4)$ \\
	для 4: $\sigma(\pi(4)) = \pi(1)$
	
	Таким образом, чтобы найти стабилизатор, осталось только перебрать все значения для подстановки $\pi$: \\
	для $\pi(1) = 1, \pi(2) = \sigma(\pi(1)) = \sigma(1) = 2, \pi(3) = \sigma(\pi(2)) = \sigma(2) = 3, \pi(4) = \sigma(\pi(3)) = \sigma(3) = 4 \Rightarrow \pi = id$ \\
	Аналогично для $\pi(1) = 2, \pi(2) = 3, \pi(3) = 4, \pi(4) = 1 \\
	\pi(1) = 3, \pi(2) = 4, \pi(3) = 1, \pi(4) = 2 \\
	\pi(1) = 4, \pi(2) = 1, \pi(3) = 2, \pi(4) = 3 \Rightarrow \\
	st(\sigma) = \{id, (1, 2, 3, 4), (1, 3)(2, 4), (1, 4, 3, 2)\}$
	
	
\end{document}
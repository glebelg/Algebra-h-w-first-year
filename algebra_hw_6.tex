\documentclass[a4paper, 12pt]{article}
\usepackage{cmap}
\usepackage[utf8]{inputenc}
\usepackage[T2A]{fontenc}
\usepackage[russian]{babel}
\usepackage[left=2cm, right=2cm, top=1cm, bottom=2cm]{geometry}
\usepackage{indentfirst}
\usepackage{amsmath, amsfonts, amsthm, mathtools, amssymb, icomma, units, yfonts}
\usepackage{amsthm}
\usepackage{algorithmicx, algorithm}
\usepackage{algpseudocode}
\usepackage{relsize}
\usepackage{graphicx}
\usepackage{tikz}
\usepackage{esvect}
\usepackage{enumerate}
\usepackage{multirow}
\usetikzlibrary{calc,matrix}
\usetikzlibrary{shapes.misc}

\makeatletter
\newenvironment{sqcases}{%
    \matrix@check\sqcases\env@sqcases
}{%
    \endarray\right.%
}
\def\env@sqcases{%
    \let\@ifnextchar\new@ifnextchar
    \left\lbrack
    \def\arraystretch{1.2}%
    \array{@{}l@{\quad}l@{}}%
}
\makeatother

\DeclareRobustCommand{\divby}{%
    \mathrel{\text{\vbox{\baselineskip.65ex\lineskiplimit0pt\hbox{.}\hbox{.}\hbox{.}}}}
}

\newcommand{\E}{\mathbb{E}}
\newcommand{\Z}{\mathbb{Z}}
\newcommand{\N}{\mathbb{N}}
\newcommand{\Q}{\mathbb{Q}}
\newcommand{\R}{\mathbb{R}}
\newcommand{\me}{\mathbbm{e}}
\newcommand{\mf}{\mathbbm{f}}
\newcommand{\Ker}{\text{Ker}}
\newcommand{\Mat}{\text{Mat}}
\newcommand{\rk}{\text{rk}}
\renewcommand{\Im}{\text{Im}}
\newcommand\tab[1][1cm]{\hspace*{#1}}
\usepackage{bbm}  % \mathbbm{}
\newcommand*\circled[1]{\tikz[baseline=(char.base)]{
        \node[shape=circle,draw,inner sep=1.5pt] (char) {#1};}}
\newcommand*\roundrect[1]{
    \begin{tikzpicture}[baseline=(char.base)]
    \node(char)[draw,fill=white,
    shape=rounded rectangle,
    minimum width=1.8cm]
    {#1};
    \end{tikzpicture}}

\begin{document}
	\title{Д/з - 6}
	\author{Чистяков Глеб, группа 167}
	\date{25 мая 2017 г.}
	
	\maketitle
	
	\textbf{№1} \\
	
	$f(x) = 2x^4-4x^3-3x^2+7x-2$
	
	$g(x) = 6x^3+4x^2-5x+1$
	
	НОД($f(x), g(x)$) - ? \\

	\begin{tikzpicture}
		\draw (0,0) node {$2x^4-4x^3-3x^2+7x-2$};
		\draw (8em,-1.5em) node {$\frac{1}{3}x-\frac{8}{9}$};
		\draw (10.5em,0em) node {$6x^3+4x^2-5x+1$};
		\draw (6em,-0.5em) -- (15em,-0.5em);
		\draw (6em,0.5em) -- (6em,-3em);
		\draw (-7em,-1.25em) -- +(0.5em,0);
		\draw (-0.65em-0.3ex,-1.5em) node {$2x^3+\frac{4}{3}x^3-\frac{5}{3}x^2+\frac{1}{3}x$} ;
		\draw (-6em,-3em) -- (4em,-3em);
		\draw (1em-0.3ex,-4em) node {$-\frac{16}{3}x^3-\frac{4}{3}x^2+\frac{20}{3}x-2$} ;
		\draw (-5em,-5em) -- +(0.5em,0);
		\draw (1em-0.3ex,-6em) node {$-\frac{16}{3}x^3-\frac{32}{9}x^2+\frac{40}{9}x-\frac{8}{9}$} ;
		\draw (-4em,-7em) -- (6.5em,-7em);
		\draw (3em-0.3ex,-8em) node {$\frac{20}{9}x^2+\frac{20}{9}x-\frac{10}{9}$} ;
	\end{tikzpicture} \\
	
	\begin{tikzpicture}
		\draw (-1.75em,0em) node {$6x^3+4x^2-5x+1$};
		\draw (7em,0em) node {$\frac{20}{9}x^2+\frac{20}{9}x-\frac{10}{9}$};
		\draw (5.5em,-2em) node {$\frac{27}{10}x-\frac{9}{10}$};
		\draw (3em,-1em) -- (11em,-1em);
		\draw (3em,0.5em) -- (3em,-3em);
		\draw (-7em,-0.9em) -- +(0.5em,0);
		\draw (-2.4em-0.3ex,-1.5em) node {$6x^3+6x^2-3x$} ;
		\draw (-6em,-2.5em) -- (1em,-2.5em);
		\draw (-0.5em-0.3ex,-3.5em) node {$-2x^2-2x+1$} ;
		\draw (-5em,-4.25em) -- +(0.5em,0);
		\draw (-0.5em-0.3ex,-5em) node {$-2x^2-2x+1$} ;
		\draw (-4em,-6em) -- (3em,-6em);
		\draw (2.5em-0.3ex,-7em) node {$0$} ;
	\end{tikzpicture} \\
	
	$\Rightarrow$ НОД($f(x), g(x)$) = $\dfrac{20}{9}x^2+\dfrac{20}{9}x-\dfrac{10}{9}$ \\
	
	Выразим его через $f(x)$ и $g(x)$:
	$$\frac{20}{9}x^2+\frac{20}{9}x-\frac{10}{9} = (2x^4-4x^3-3x^2+7x-2) - (6x^3+4x^2-5x+1)\cdot(\frac{1}{3}x-\frac{8}{9})$$ \\
	
	\textbf{№2} \\
	
	1) В кольце $\mathbb{C}$: \\
	$x^6+x^3-12 = (x^3+4)(x^3-3)$ \\
	
	$\bullet$ $x^3+4 = 0 \Rightarrow x = \sqrt[3]{-4}$
	
	$-4 = |-4 + i\cdot0| = \sqrt{(-4)^2} = 4 \Rightarrow$
	$$-4 = 4(cos(\pi) +i\cdot sin(\pi))$$
	$$\sqrt[3]{-4} = \sqrt[3]{4}(cos(\dfrac{\pi+2\pi k}{3}) + i\cdot sin(\dfrac{\pi+2\pi k}{3})$$
	где $k = 0, 1, 2$ \\
	\tab $k = 0$: $x_1 = \sqrt[3]{4}(cos(\dfrac{\pi}{3}) + i\cdot sin(\dfrac{\pi}{3}) = \dfrac{\sqrt[3]{4}}{2} + \dfrac{\sqrt[3]{4}\cdot\sqrt{3}\cdot i}{2}$ \\
	\tab $k = 1$: $x_2 = \sqrt[3]{4}(cos(\pi) + i\cdot sin(\pi) = -\sqrt[3]{4}$ \\
	\tab $k = 2$: $x_3 = \sqrt[3]{4}(cos(\dfrac{5\pi}{3}) + i\cdot sin(\dfrac{5\pi}{3}) = \dfrac{\sqrt[3]{4}}{2} - \dfrac{\sqrt[3]{4}\cdot\sqrt{3}\cdot i}{2}$ \\
	
	$\bullet$ $x^3-3 = 0 \Rightarrow x = \sqrt[3]{3}$
	
	$3 = |3 + i\cdot0| = \sqrt{3^2} = 3 \Rightarrow$
	$$3 = 3(cos(2\pi) +i\cdot sin(2\pi))$$
	$$\sqrt[3]{3} = \sqrt[3]{4}(cos(\dfrac{2\pi+2\pi k}{3}) + i\cdot sin(\dfrac{2\pi+2\pi k}{3})$$
	где $k = 0, 1, 2$ \\
	\tab $k = 0$: $x_4 = \sqrt[3]{3}(cos(\dfrac{2\pi}{3}) + i\cdot sin(\dfrac{2\pi}{3})) = -\dfrac{\sqrt[3]{3}}{2} + \dfrac{\sqrt[3]{3}\cdot\sqrt{3}\cdot i}{2}$ \\
	\tab $k = 1$: $x_5 = \sqrt[3]{3}(cos(\dfrac{4\pi}{3}) + i\cdot sin(\dfrac{4\pi}{3})) = -\dfrac{\sqrt[3]{3}}{2} - \dfrac{\sqrt[3]{3}\cdot\sqrt{3}\cdot i}{2}$ \\
	\tab $k = 2$: $x_6 = \sqrt[3]{3}(cos(2\pi) + i\cdot sin(2\pi)) = \sqrt[3]{3}$ \\
	
	$\Rightarrow x^6+x^3-12 = (x - (\dfrac{\sqrt[3]{4}}{2} + \dfrac{\sqrt[3]{4}\cdot\sqrt{3}\cdot i}{2}))(x + \sqrt[3]{4})(x - (\dfrac{\sqrt[3]{4}}{2} - \dfrac{\sqrt[3]{4}\cdot\sqrt{3}\cdot i}{2}))(x + (\dfrac{\sqrt[3]{3}}{2} - \dfrac{\sqrt[3]{3}\cdot\sqrt{3}\cdot i}{2}))(x + (\dfrac{\sqrt[3]{3}}{2} + \dfrac{\sqrt[3]{3}\cdot\sqrt{3}\cdot i}{2}))(x - \sqrt[3]{3})$ \\
	
	2) В кольце $\R$: \\
	$(x - (\dfrac{\sqrt[3]{4}}{2} + \dfrac{\sqrt[3]{4}\cdot\sqrt{3}\cdot i}{2}))(x + \sqrt[3]{4})(x - (\dfrac{\sqrt[3]{4}}{2} - \dfrac{\sqrt[3]{4}\cdot\sqrt{3}\cdot i}{2}))(x + (\dfrac{\sqrt[3]{3}}{2} - \dfrac{\sqrt[3]{3}\cdot\sqrt{3}\cdot i}{2}))(x + (\dfrac{\sqrt[3]{3}}{2} + \dfrac{\sqrt[3]{3}\cdot\sqrt{3}\cdot i}{2}))(x - \sqrt[3]{3}) = (x + \sqrt[3]{4})(x^2 - \sqrt[3]{4}x + 2\sqrt[3]{2})(x - \sqrt[3]{3})(x^2 + \sqrt[3]{3}x + \sqrt[3]{9}) \Rightarrow$ \\
	$$x^6+x^3-12 = (x + \sqrt[3]{4})(x^2 - \sqrt[3]{4}x + 2\sqrt[3]{2})(x - \sqrt[3]{3})(x^2 + \sqrt[3]{3}x + \sqrt[3]{9})$$ \\
	
	\textbf{№3} \\
	
	$5+\sqrt{-5}$ -- простое, если оно необратимо и $\nexists$ таких необратимых $x,y \in \Z[\sqrt{-5}]$:\\ $5+\sqrt{-5} = xy$. Для начала проверим $5+\sqrt{-5}$ на обратимость: \\
	Знаем, что $5+\sqrt{-5}$ -- обратима, если $\exists z\in \Z[\sqrt{-5}]: (5+\sqrt{-5})\cdot z = z\cdot (5+\sqrt{-5}) = 1$. Пусть $z = a + i\cdot b\sqrt{5}$, тогда (т.к. $5+\sqrt{-5} = 5+i\cdot\sqrt{5}$):
	$$(5+i\cdot\sqrt{5})(a + i\cdot b\sqrt{5}) = 5a + i\cdot5b\sqrt{5} +i\cdot a\sqrt{5}-5b = 1$$
	$\begin{cases}
		5a-5b = 1 \\
		5b\sqrt{5}+a\sqrt{5} = 0 \Rightarrow a = -5b
	\end{cases}$
	 $\Rightarrow -25b-5b = 1 \Rightarrow b = -\dfrac{1}{30}, a = \dfrac{1}{6}$ \\
	 Так как $a, b \notin \Z$, то $5+\sqrt{-5}$ -- необратимо. \\
	 
	 Возьмем $x = a + i\cdot b\sqrt{5}, y = c + i\cdot d\sqrt{5}$ и проверим, существуют ли такие $a, b, c, d \in\Z$, что $5+\sqrt{-5} = xy$:
	 $$5+\sqrt{-5} = xy = (a + i\cdot b\sqrt{5})(c + i\cdot d\sqrt{5}) = ac + i\cdot ad\sqrt{5} + i\cdot bc\sqrt{5} - 5bd$$
	 $\begin{cases}
	 ac - 5bd = 5 \\
	 ad\sqrt{5} + bc\sqrt{5} = \sqrt{5}
	 \end{cases}$ 
	 Возьмем $a = 0, b = 1 \Rightarrow c = 1, d = -1$. Тогда наши $x$ и $y$ выглядят так: $x = i\cdot \sqrt{5}, y = 1 - i\cdot\sqrt{5}$. Проверим их на обратимость: \\
	 $\bullet$ $i\cdot \sqrt{5}\cdot(f + i\cdot g\sqrt{5}) = i\cdot f\sqrt{5} - 5g = 1 \Rightarrow$
	 $\begin{cases}
	 	-5g = 1 \\
	 	i\cdot f\sqrt{5} = 0
	 \end{cases}$
	 $\Rightarrow g \notin\Z\Rightarrow x$ -- необратим. \\
	 $\bullet$ $(1 - i\cdot\sqrt{5})(f + i\cdot g\sqrt{5}) = f + i\cdot g\sqrt{5} - i\cdot f\sqrt{5} + 5g = 1 \Rightarrow$
	 $\begin{cases}
	 f + 5g = 1 \Rightarrow f = 1 - 5dg \\
	 g\sqrt{5} - f\sqrt{5} = 0
	 \end{cases}$
	 $\Rightarrow 6g - 1 = 0 \Rightarrow g \notin\Z\Rightarrow y$ -- необратим. \\
	 Таким образом $5+\sqrt{-5} = xy$, где $x$ и $y$ необратимые, а следовательно $5+\sqrt{-5}$ -- не простое. \\
	 
	 \textbf{№4} \\
	 
	 Предположим, что $N$ принимает конечное число значений. Тогда выберем такой элемент $x$, что $N(x)$ -- максимально. И еще возьмем необратимый элемент $y$, тогда, по определению нормы, $N(xy) \geqslant N(x)$. Но равенства там не может быть, так как $y$ -- необратимый (по лемме: Пусть R -- евклидово кольцо и $a, b \in R\setminus \{0\}$. Тогда $N(ab) = N(a) \Leftrightarrow b$ -- обратим) $\Rightarrow N(xy) > N(x)$. Так как $xy \in R$ $\Rightarrow$ что существует элемент кольца с большей нормой $\Rightarrow$ наше предположение неверно, а значит $N$ принимает бесконечное число значений. \\
	
\end{document}